\chapter{Qualité}
La qualité d'un programme est une grande valeur ajoutée à tout projet. Les critères qualités considérés pour ce projet sont les suivants :
\begin{itemize}
  \item les tests. Effectivement, les tests permettent de certifier certains comportement du programme.
  Nous avons choisi des tests à granularité faible (tests unitaires) avec des assertions.
  \item analyse du code.
  \item gestionnaire de projet.
  \item une documentation.
\end{itemize}


\section{Tests}

Les tests ont été effectués à l'aide de deux outils. \\
Le premier est la bibliothèque de tests cppunit. 
Celle-ci offre diverses méthodes pour lancer des pipelines de tests. 
Nous avons réaliser des tests dits de Boite Noire avec cette bibliothèque. 
Le but est que ces tests ne soient pas intrusifs dans le code.
Pour utiliser les tests, il suffit de saisir la commande \emph{make check} à la racine du code.

\subsection{Tests de persistances}
Le minimum est d'effectuer des tests de persistances pour chaque algorithme. 
En effet, nous utilisons des graines de générations ainsi que des génénateurs pseudo-aléatoires. \\
La stratégie de comparaison est la suivante :
\begin{itemize}
 \item préparation du test \textit{via} la méthode setUp(). 
 Il s'agit de construire tous les objets communs aux différents tests. 
 Il est évident que les objets initialisés sont corrects car la logique admet que Faux implique n'importe quoi.
 \item appel des différentes méthodes de tests \textit{via} les macros de cppunit. 
 Chaque macro appelle une fonction test qui exécute elle-même un test logique (Assertion).
 Ce dernier compare les hauteurs de deux objets heightmap point par point. 
 \item terminaison du test avec la méthode tearDown().
\end{itemize}

\subsection{Tests de l'existant}
La cohérence de nos résultats ont été vérifiés en les comparant avec un équivalent de l'existant.
Pour le bruit de Perlin, nous postulons que les résultats sont cohérents. Effectivement,
ces deux méthodes de génération sont issues de LibNoise.
Pour les autres bruits et méthodes fractales, il existe un grand nombre de variantes donc il est difficile
de comparer le résultat obtenu avec celui existant. Néanmoins, il a été possible d'établir des 
critères de Résultats Attendus en se basant sur les caractéristiques principales des bruits :
\begin{itemize}
 \item bruit de simplex: il phénomène de répétition à grande échelle 
 avec des regroupements de pixels de hauteurs équivalentes;
 \item érosion thermique et hydraulique : apparition d'un flou en bordure des hauteurs avec des variations de hauteurs à ces niveaux;
 \item midpoint displacement et diamond square : les heightmap sont très structurés avec des dégradés très doux 
 ce qui est aussi visible sur d'autres versions de midpoint displacement et diamond square.
\end{itemize}

\section{Analyse du code}

\subsection{Statique}
CppCheck est sous Licence GPL et est suffisant pour l'analyse statique du code que nous désirons faire.
Cet outil permet de repérer les erreurs de conception de chacune des classes non détectable par 
le compilateur g++. Ainsi, il aide à développer une architecture plus propre et plus cohérente.

\subsection{Dynamique}
Valgrind est un profileur de code. Il nous a servi à détecter les fuites mémoires présents dans le code.

\section{Gestionnaire de projet}
Un projet fonctionnel passe par une gestion rigoureuse du code et un système de communication efficace.
\subsection{Gestionnaire de version}
Gît et Subversion ont été tous les utilisés lors de ce projet. Gît pour toutes les révisions et 
Svn, pour des versions stables, lors du partage avec les encadrant du projet.
\subsection{Communication}
Trello est couramment utilisé dans le cadre de projet en groupe restreint. Il se base sur un système
de Post-It avec des assignations et des dates.
Il nous a servi à planifier les tâches (code, rapport et réunions).


\section{Documentation}
La documentation permet une meilleur compréhension du code et de l'architecture logicielle.
Nous avons utilisé le gestionnaire de documentation Doxygen car celui-ci est sous Licence GPL
et est performant. Toutes les classes, méthodes et attributs du projet (ainsi que les tests) ont été commentés.
Pour générer la documentation, il faut réaliser un \emph{make doc}.


