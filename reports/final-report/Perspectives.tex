\chapter{Perspectives}
Les délais ne nous ont pas permis de mener à terme les objectifs que nous nous étions fixé. 
Aussi il sera possible d'apporter d'autres amélioration au programme.

\section{Atteindre les objectifs fixés}

\subsection{Prioritaires}

\subsubsection{Algorithmes}
Certains algorithmes n'ont pas été implémentés comme Fractional Brownian Motion. Même si ce dernier est 
simple, nous nous sommes concentrés sur les autres algorithmes tels que les érosions. 
Pour produire Fractional Brownian Motion, il sera possible d'utiliser certains modules de LibNoise
(module d'ajout, etc). Et cela sera possible car nos bruits héritent de LibNoise (\textit{via} NoiseBase).
\subsubsection{Maillage et sphère}
Les maillages triangulaires et hexagonaux ne sont pas présents mais il serait intéressant de les
implémentés. Les sphères pourraient être conçues en partant d'un carte cubique ou d'un isocahèdre.
\subsubsection{Raffinement locale}
Un raffinement existe mais il ne permet pas de zoomer sur une zone de la carte. Il faudrait 
utiliser les Quad-Tree pour cela. Chaque noeud du Quad-Tree contiendrait un face avec les vertex de la grille.
Chaque fils de la grille serait une partie de la grille subdivisée (tesselation des grilles). Ainsi, il serait possible d'obtenir :
\begin{itemize}
 \item quatre fils pour les grilles carrées et triangulaires;
 \item sept fils pour les grilles hexagonales.
\end{itemize}

\subsection{Secondaires}

\subsubsection{Implémentation d'un visualisateur}
Il serait intéressant d'implémenter le visualisateur proposé sur le cahier des charges. De plus, 
l'intégration avec le Quad-Tree permettrait d'obtenir une génération procédurale temps-réel avec des fonctions de zoom.

\section{Amélioration des tests}
Les tests unitaires effectués, jusqu'à présent, ne couvrent que les méthodes de générations. Aussi, 
les grilles carrées, pour les cartes d'élévation planaires, n'ont pas été testées de façon approfondi. 
Aussi, les exportations pour Blender3D sont à tester correctement. Effectivement, nous nous sommes uniquement assurés 
de la lisibilité des fichiers exportés sur Blender3D.

\section{Diminution du temps de calcul}
Avec l'intégration d'un visualisateur, il va nécessairement falloir gagner du temps sur le calcul pour avoir un rendu temps-réel.
Ainsi, l'utilisation du GPU est préférable au CPU. Il faudrait utiliser CUDA et/ou les shaders d'OPENGL pour effectuer
les calculs sur le GPU.

\section{Interopérabilité}
L'un des concept de ce projet est l'importation et l'exportation des cartes d'élévation. En allant plus loin,
il serait possible de :
\begin{itemize}
 \item adapter les cartes d'élévation pour d'autres logiciels;
 \item proposer des classes de conversions de cartes d'élévation entre logiciels;
 %%TODO\item .  
\end{itemize}
