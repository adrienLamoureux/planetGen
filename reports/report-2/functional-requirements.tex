%%terragen :  Donc, un fois que nous aurons généré une carte à l'aide de notre bibliothèque, nous devrons réaliser une conversion au format .TER afin qu'elle soit compatible avec terragen.


\chapter{Besoins fonctionnels}

\section{Besoins prioritaires}

Ces besoins sont ceux que nous nous engageons à implémenter.

\subsection{Structures de données}
La bibliothèque devra proposer deux types de cartes d'élévations :\\
\textbf{sphériques} et \textbf{carrées}.

\subsection{Génération d'un terrain}
\subsubsection{Algorithmes}
Pour permettre la génération d'un terrain, les méthodes suivantes devront pouvoir
 être appliquées sur les cartes d'élévation (les méthodes suivies de la mention [prioritaire] sont celles qui doivent \^etre implémentées en priorité) :\\

\noindent méthodes des fractales :
\begin{itemize}
\item diamond-square ; [prioritaire]
\item midpoint displacement ; [prioritaire]
\item fractional brownian motion ;
\item multi-fractal midpoint displacement.
\end{itemize}
méthodes des bruits :
\begin{itemize}
\item simple noise ;
\item cell noise ;
\item Perlin noise ; [prioritaire]
\item value noise.
\end{itemize}
raffinement par modèles :
\begin{itemize}
\item erosion model.
\end{itemize}

\subsubsection{Raffinement}
S'il souhaite raffiner une zone précise, l'utilisateur pourra la sélectionner (en passant ses coordonnées en paramètres) afin d'appliquer une des méthodes de raffinement (de son choix) sur cette seule zone.

%\subsubsection{Risque}
%Biaiser la cohérence du terrain au niveau de la bordure de la zone raffinée.

%\subsubsection{Parade}
%Proposer une zone de lissage autour de la frontière de la zone raffinée.

%\subsubsection{Validation}
%Stresser la bibliothèque en la soumettant à divers facteurs :
%\begin{itemize}
%\item variation des paramètres d'entrée des méthodes de générations procédurales de façon aléatoire ;
%\item diverses tailles de cartes sous différents niveaux de détails ;
%\item exécution sur diverses machines ayant des différences au niveau physique (RAM, CPU, etc).
%\end{itemize}

\subsubsection{Génération de texture}
Il devra être possible de donner une couleur différente à chaque point de la carte en
fonction de son élévation.

\subsection {Génération d'une grille}

Pour les cartes carrées et sphériques, il sera possible de choisir une grille carrée, triangulaire ou hexagonale.

\subsection{Importation / exportation}
Les cartes d'élévations pourront être importées et exportées sous forme
d'images (.png, .jpeg) afin d'être lisible par d'autres logiciels
tels que Terragen 3 ou Blender 3D. \newline
Les modèles sont exportés sous format .obj pour être lisible par Blender3D.
Une méthode sera donc conçue pour lire les fichiers importés et une autre pour les concevoir dans le cas d'un export. 

\section{Besoins non prioritaires}

\subsection{Implémentation d'un visualisateur 3D}
Génération d'une simple représentation de la carte par rapport à la
carte d'élévation d'origine et du modèle construit par la bibliothèque.

Implémentation d'un zoom pour pouvoir visualiser la carte avec différents niveaux
de détail.
